\section{Ejemplos de programación lineal}

%\lstinputlisting[language=Octave]{BitXorMatrix.m}
%\lstinputlisting[language=python]{../codigo/programacion-lineal/planeacion_produccion.py}

\subsection{Problema de proyectos de
	inversión}\label{problema-de-proyectos-de-inversiuxf3n}

Imaginemos que ocupamos el puesto de coordinador de proyectos dentro de
una empresa. El gerente general de dicha empresa ha destinado 100,000
pesos para invertir en los proyectos que generen beneficios económicos a
esta. Existen tres proyectos en los que se puede invertir. ¿En cuál(es)
proyecto(s) debería invertir la empresa para obtener los máximos
beneficios económicos?

Se tiene la siguiente información sobre los proyectos:

\begin{longtable}[]{@{}ccl@{}}
\toprule
Nombre & Costo de Inversión & Benefició económico \\
\midrule
\endhead
Proyecto A & \$50,000 & \$80,000 \\
Proyecto B & \$70,000 & \$90,000 \\
Proyecto C & \$25,000 & \$30,000 \\
\bottomrule
\end{longtable}

\lstinputlisting[caption= Proyectos de inversión]{../codigo/introduccion/proyectos-inversion/proyectos_inversion.py}

\subsection{Fabricación de ropa}\label{fabricaciuxf3n-de-ropa}

Una costurera fabrica y vende faldas y pantalones de mezclilla, para lo
cual cada semana compra un rollo de 50 metros de mezclilla. Para hacer
un pantalón requiere 2 metros de tela, mientras que para una falda, 1.5
metros.

Por lo general, ella trabaja ocho horas diarias, de lunes a viernes.
Para hacer un pantalón requiere tres horas, mientras que hacer una falda
le toma una. Un pantalón le genera 80 pesos de ganancia, mientras que al
vender una falda gana 50 pesos.

Construir un modelo matemático que permita maximizar la ganancia semanal
de la costurera, considerando que todo producto que fabrique puede
venderlo.

\lstinputlisting[caption= Proyectos de inversión]{../codigo/introduccion/costura/costura.py}

    \hypertarget{problemas-de-mezcla-de-productos}{%
	\subsection{Problemas de mezcla de
		productos}\label{problemas-de-mezcla-de-productos}}

Una compañía fabrica tres productos: crema corporal, crema facial y
crema para bebés. Los tres productos comparten ingredientes en su
elaboración: mezcla base, aceite de almendras, vitamina E y manteca de
karité. En la siguiente tabla se presenta información acerca de los
porcentajes de composición de cada uno de los tres productos:

\begin{longtable}[]{@{}lllll@{}}
\toprule
. & Mezcla base & Aceite de Almendras & Vitamina E & Manteca de
karité \\
\midrule
\endhead
Crema Corporal & 90\% & 4\% & 1\% & 5\% \\
Crema facial & 85\% & 8\% & 2.5\% & 4.5\% \\
Crema para bebé & 80\% & 10\% & - & 10\% \\
\bottomrule
\end{longtable}

Cada día, la compañía cuenta con 500 litros de la mezcla base, 50 litros
de aceite de almendras, 5 litros de vitamina E y 30 litros de manteca de
karité. Adicionalmente, se tiene la siguiente información sobre costos y
precios de venta.

\begin{longtable}[]{@{}ll@{}}
\toprule
Ingrediente & Costo por litro \\
\midrule
\endhead
Mezcla base & \$20 \\
Aceite de almedras & \$500 \\
Vitamina E & \$1500 \\
Manteca de karité & \$200 \\
\bottomrule
\end{longtable}

\begin{longtable}[]{@{}ll@{}}
\toprule
Producto & Precio de venta (\$/L) \\
\midrule
\endhead
Crema corporal & \$80 \\
Crema facial & \$120 \\
Crema para bebé & \$100 \\
\bottomrule
\end{longtable}

La demanda diaria de la crema corporal es de 200 litros; de la crema
facial, 150 litros; y de la crema para bebé, de 250 litros. Por
políticas de la empresa, se deben fabricar al menos 50 litros de crema
facial. ¿Cuánto de cada producto deberá producir la compañía para
maximizar su utilidad?

\lstinputlisting[caption= Proyectos de inversión]{../codigo/introduccion/mezcla-productos/mezcla_productos.py}