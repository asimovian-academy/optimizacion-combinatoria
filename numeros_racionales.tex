
%102
\section{Los números racionales}


	Los números racionales son el conjunto de números
	$$
	\Q=\set{\frac{a}{b} \mid a,b \in \Z, b\neq 0} 
	$$
identificando $\dfrac{a}{b}=\dfrac{c}{d}$ siempre que la \emph{razón cruzada} sea igual
\begin{align*}
ad=bc
\end{align*}


%%%%%%%%%%%%%%%%%%%%%


	¿Para que nos sirve $\Q$? 
	
	Este conjunto de números nos sirve para \emph{contar, sumar, restar, multiplicar y dividir.}



	\begin{defn}
		\label{rat:equiv}
		Dos números racionales $\dfrac{a}{b},\dfrac{c}{d}$ son equivalentes si
		$$
		ad-bc=0.
		$$
	\end{defn}

%%%%%%%%%%%%%%%%%%%%%
{}

	\begin{problema}
		$\dfrac{1}{2}$ es equivalente a $\dfrac{2}{4}$ porque
		$$
		\left( 1 \right)\left( 4 \right)-\left( 2 \right)\left( 2 \right)=0.
		$$
	\end{problema}
	


\subsection{Simplificaci\'on}

 
	\begin{defn}
		\begin{enumerate}
			\item Diremos que dos enteros $p,q\in \Z$ son primos relativos si $\texttt{mcd}(p,q)=1.$ 
			\item Diremos que $\dfrac{p}{q}\in \Q$ es la forma \emph{irreducible} de $\dfrac{a}{b}\in \Q$ si 
			\begin{itemize}
				\item $\dfrac{p}{q}$ es equivalente a $\dfrac{a}{b}$ y
				\item $p,q$ son primos relativos.
			\end{itemize}
			
		\end{enumerate}
		
	\end{defn}
	




	Por la definici\'on \ref{rat:equiv}, tenemos que para un número racional $\frac{a}{b}:$
	$$
	\dfrac{n\cdot a}{n\cdot b}=\dfrac{a}{b},
	$$
	siempre que $n\neq 0.$
	
	\begin{problema}
		$\dfrac{2}{4}=\dfrac{2*1}{2*2}=\dfrac{1}{2}.$
	\end{problema}
	

% 
% 
%  \begin{defn}
%   Diremos que la forma irreducible de una fracci\'on $\dfrac{a}{b}$ es $\dfrac{p}{q}$ si
%   \begin{itemize}
%    \item $$\dfrac{a}{b}=\dfrac{p}{q},$$ pero
%    \item $p,q$ son primos relativos.
%   \end{itemize}
% 
%  \end{defn}
% 
% 


	\begin{rem}
		Sean $a,b$ dos enteros positivos.
		Si $d=\texttt{mcd}(a,b)$ y $$a=d\cdot p, \, b=d\cdot q,$$ entonces podemos simplificar de la siguiente manera
		$$
		\dfrac{a}{b}=\dfrac{d\cdot p}{d \cdot q}=\dfrac{p}{q}.
		$$
	\end{rem}




	\begin{rem}
		Si $d$ es el máximo común divisor de los enteros $a,b\neq 0,$ entonces tenemos que $p,q$ son los cocientes en las operaciones
		$$\begin{cases}
			a=d\cdot p \\
			b=d\cdot q
		\end{cases}
		$$
	\end{rem}
	



	\begin{problema}
		Encuentre la forma irreducible de $\dfrac{15}{10}.$
	\end{problema}



	
	\begin{solucion}
		\begin{enumerate}
			\item Primero, muestre que $\texttt{mcd}(15,10)=5;$ 
			\item Como $$\dfrac{15}{10}=\dfrac{5\cdot 3}{5\cdot 2}=\dfrac{3}{2},$$ entonces $\dfrac{3}{2}$ es equivalente a 
			$\dfrac{15}{10}$ 
			\item Finalmente muestre que $3$ y $2$ son primos relativos. Concluimos que $\dfrac{3}{2}$ es la forma irreducible de 
			$\dfrac{15}{10}.$ 
		\end{enumerate}
		
	\end{solucion}



	\begin{problema}
		Encuentre la forma irreducible de la fracci\'on $$\dfrac{182}{910}$$
	\end{problema}
	



\subsection{Conversi\'on y comparaci\'on}


	Supongamos que una pizza se parte en 12 rebanadas iguales, mientras que otra similar se parte en 8. ¿Qu\'e cantidad de pizza es mayor, 7 rebanadas de la primera o 5 de la segunda?



	Para comparar dos fracciones, debemos convertirlas de manera que tenga un común denominador. 



	\begin{alg}[Conversi\'on a común denominador]
		Para convertir dos fracciones $\frac{a}{b}, \frac{c}{d}$ a común denominador:
		\begin{enumerate}
			\item Encuentre $m=\texttt{mcm}(b,d)$
			\item Encuentre un entero $p$ tal que $m=b\cdot p$ y convierta la primera fracci\'on
			$$
			\dfrac{a}{b}=\dfrac{a\cdot p}{b \cdot p}=\dfrac{ap}{m}
			$$
			\item Encuentre un entero $q$ tal que $m=d\cdot q$ y convierta la segunda fracci\'on
			$$
			\dfrac{c}{d}=\dfrac{c\cdot q}{d \cdot q}=\dfrac{cq}{m}
			$$
		\end{enumerate}
		
	\end{alg}
	



	\begin{rem}
		Si el común demoninador $m$ de dos fracciones $$\dfrac{x}{m}, \dfrac{y}{m}$$ es positivo, entonces
		$$
		\dfrac{x}{m} < \frac{y}{m} \iff x < y.
		$$
	\end{rem}
	




	\begin{problema}
		Compare cada uno de los siguientes pares de fracciones:
		\begin{enumerate}
			\item $\dfrac{15}{11}, \dfrac{28}{37}$ 
			\item $-\dfrac{35}{36}, \dfrac{1}{6}$
			\item $\dfrac{3}{10}, -\dfrac{23}{33}$
			\item $-\dfrac{17}{31}, -\dfrac{12}{7}$
		\end{enumerate}
		
	\end{problema}


\subsection{Operaciones}


	\begin{alg}[Suma de fracciones]
		Para sumar dos fracciones $\frac{a}{b}, \, \frac{c}{d}:$
		\begin{enumerate}
			\item Convierta a común denominador, de manera que
			$$\dfrac{a}{b}=\dfrac{x}{m}, \, \dfrac{c}{d}=\dfrac{y}{m};$$ 
			\item sume ambos numeradores
			$$
			\dfrac{a}{b}+\dfrac{c}{d}=\dfrac{x}{m}+\frac{y}{m}=\dfrac{x+y}{m};
			$$
			\item simplifique.
		\end{enumerate}
		
	\end{alg}
	




	%  La suma entre dos números racionales se define como
	%  \begin{equation}
	%   \label{rat:sum}
	%   \dfrac{a}{b}+\dfrac{c}{d}=\dfrac{ad+bc}{bc}.
	%  \end{equation}
	% 
	\begin{problema}
		$$
		\dfrac{2}{3}+\dfrac{4}{5}=%\dfrac{(2)(5)+(3)(4)}{(3)(5)}=\dfrac{22}{15}
		$$
	\end{problema}
	



	\begin{rem}
		Cualquier suma se puede reescribir como una resta:
		$$x+y=x-\left( -y \right),$$ 
		y viceversa
		$$x-y=x+\left( -y \right).$$ 
		
		Por esta raz\'on, en álgebra, no es muy útil distinguir entre estas dos operaciones. Utilizaremos el mismo algoritmo, para encontrar la resta de dos fracciones. 
	\end{rem}
	




	%  La resta entre dos números racionales se define como
	%  \begin{equation}
	%   \label{rat:rest}
	%   \dfrac{a}{b}-\dfrac{c}{d}=\dfrac{ad-bc}{bc}.
	%  \end{equation}
	% 
	\begin{problema}
		$$
		\dfrac{2}{3}-\dfrac{4}{5}=%\dfrac{(2)(5)-(3)(4)}{(3)(5)}=-\dfrac{2}{15}
		$$
	\end{problema}
	



	\begin{problema}
		Realice las siguientes y escriba el resultado en su forma irreducible:
		\begin{enumerate}
			\item $\dfrac{5}{3}+\dfrac{5}{9}$
			\item $\dfrac{7}{3}+\dfrac{4}{7}$
			\item $\dfrac{5}{4}+\dfrac{3}{2}$
			\item $\dfrac{1}{2}-\dfrac{1}{3}$
			\item $\dfrac{3}{5}-\dfrac{4}{9}$
		\end{enumerate}
		
	\end{problema}
	

%%%%%%%%%%%%%%%%%%%%%% 

	
	La multiplicaci\'on entre dos números racionales se define como
	\begin{equation}
		\dfrac{a}{b}\cdot\dfrac{c}{d}=\dfrac{a\cdot c}{b\cdot d}.
	\end{equation}
	
	\begin{problema}
		$\dfrac{2}{3}\cdot \dfrac{4}{5}=\dfrac{2\cdot 4}{3\cdot 5}=\dfrac{8}{15}.$
	\end{problema}
	




	\begin{rem}
		En ocasiones, la divisi\'on de fracciones se conoce como regla del ``sandwich'':
		$$
		\dfrac{\left( \dfrac{a}{b} \right)}{\left( \dfrac{c}{d} \right)}=\dfrac{a}{b}\div \dfrac{c}{d}=\dfrac{a\cdot d}{b \cdot c}
		$$
	\end{rem}
	




	
	La divisi\'on entre dos números racionales se define como
	\begin{equation}
		\dfrac{a}{b}\div\dfrac{c}{d}=\dfrac{a\cdot d}{b\cdot c}.
	\end{equation}
	siempre y cuando $c\neq 0.$
	 
	
	\begin{problema}
		$\dfrac{2}{3}\div \dfrac{4}{5}=\dfrac{2\cdot 5}{3\cdot 4}=\dfrac{10}{12}.$
	\end{problema}
	

